\documentclass[a4paper,12pt]{article}
\usepackage[russian]{babel}
\usepackage[utf8]{inputenc}
\usepackage{array}
\usepackage{multirow}
\usepackage{multicol}
\usepackage{makecell}

\usepackage[a4paper,top=1.5cm,bottom=1.5cm,left=1.5cm,right=1.5cm]{geometry}


\begin{document}
\setlength{\tabcolsep}{6pt} % Уменьшаем горизонтальные отступы
\fontsize{8}{12}\selectfont
\begin{multicols}{2}
    \begin{tabular}{c c c c c c c c c}
        Участник & \multicolumn{6}{c}{Баллы по задачам} & Сумма & Медаль \\ 
        & 1 & 2 & 3 & 4 & 5 & 6 & & \\ 
        \makecell[{{p{1.5cm}}}]{\raggedright Григорьев\\Михаил} & 7 & 7 & 5 & 7 & 7 & 0 & 33 & золотая \\ 
        \makecell[{{p{1.5cm}}}]{\raggedright Кальмынин\\Александр} & 7 & 7 & 0 & 7 & 7 & 0 & 28 & золотая \\ 
        \makecell[{{p{1.5cm}}}]{\raggedright Клоев\\Даниил} & 7 & 7 & 7 & 7 & 7 & 0 & 35 & золотая \\ 
        \makecell[{{p{1.5cm}}}]{\raggedright Крачун\\Дмитрий} & 7 & 7 & 3 & 7 & 7 & 1 & 32 & золотая \\ 
        \makecell[{{p{1.5cm}}}]{\raggedright Матушкин\\Александр} & 7 & 7 & 3 & 7 & 1 & 1 & 26 & серебряная \\ 
        \makecell[{{p{1.5cm}}}]{\raggedright Шабанов\\Лев} & 7 & 0 & 3 & 6 & 0 & 7 & 23 & серебряная \\ 
    \end{tabular}
    Руководители команды благодарят Д.Ю.Дойхена, который много лет оказывает поддержку команде России в международных математических соревнованиях   
    1. Дан треугольник ABC; точка J является центром вневписанной окружности, соответствующей вершине A. Эта вневписанная окружность касается отрезка BC в точке M, а прямых AB и AC – в точках K и L соответственно. Прямые LM и BJ пересекаются в точке F, а прямые KM и CJ пересекаются в точке G. Пусть S – точка пересечения прямых AF и BC, а T – точка пересечения прямых AG и BC. Докажите, что точка M является серединой отрезка ST. Греция
    \columnbreak
    усмотрению множество S, состоящее из целых положительных чисел (возможно, это множество уже было указано в одном из предыдущих вопросов), и спрашивает игрока A, принадлежит ли число x множеству S. Игрок B может задать столько вопросов, сколько он хочет. На каждый вопрос игрока B игрок A должен сразу ответить «да» или «нет», при этом ему разрешается соврать столько раз, сколько он хочет; единственное ограничение состоит в том, что из любых k + 1 подряд идущих ответов хотя бы один ответ должен быть правдивым. После того как B задаст столько вопросов, сколько он сочтет нужным, он должен указать множество X, содержащее не более n целых положительных чисел. Если x принадлежит множеству X, то игрок B выиграл; иначе B проиграл. Докажите, что:
    \end{multicols}


\end{document}
